\section{Underlying Notions}
{\centering \textbf{We are going to adopt the convention that every map is continuous unless otherwise stated.}}\\

\noindent
We want to be able to talk about ways to transform some shapes into other shapes and think about them then being equal. We do this by considering an $I \in [0, 1]$ that we can paramaterise a function by. So if we have two curves $s : X \to X$ and $t : X \to X$, we can define some $f : X \times I \to X$ such that,
$$ \begin{cases}
  f_0(x) &= s(x) \\
  f_1(x) &= t(x)
\end{cases} $$

\begin{ndefi}[Deformation Retraction]
  A Deformation retraction of a space $X$ onto an $A \subset X$ is a family of maps, $f_t : X \to X$ $t \in I$, such that, $f_0 = \1$, $f_t(X) = A$ and $f_t | A = \1$ $\forall t$. The family $f_t$ should be continuous in the sense that the associated map $X \times I \to X$, $(x, t) \mapsto f_t(x)$ is continuous.
\end{ndefi}

\begin{ndefi}[Mapping Cylinder]
  A mapping cylinder, $M_f$, for a $f : X \to Y$ is the quotient space of the disjoint union $(X \times I)\coprod Y$, obtained by identifying each $(x, 1) \in X \times I$ with $f(x) \in Y$.
\end{ndefi}

A mapping cylinder deformation, of $f$, retracts to $Y$.\\

A deformation retraction $f_t : X \to X$ is just a special case of a homotopy, which is just simply any family of maps, $f_t : X \times I \to Y$, given by $F(x, t) = f_t (x)$ is continuous.

\begin{ndefi}[Homotopic]
  One says two maps are homotopic if there exists a homotopy $f_t$ connecting them and one writes $f_0 \hm f_1$
\end{ndefi}

\textbf{Retractions are the topological analogue of projection operators.}

\begin{ndefi}[Homotopy Relative]
  A homotopy $f_t : X \to Y$ whose restriction to a subspace $A \subset X$ is independent of $t$ is homotopy relative, written as homotopy rel.
\end{ndefi}


\begin{ndefi}[Homotopy Equivalence]
  A map $f : X \to Y$ is called a homotopy equivalence if there exists a $g : Y \to X$ such that $fg \hm \1$ and $gf \hm \1$.
\end{ndefi}

Moreover, we can now say that $X$ and $Y$ are homotopy equivalent or have the same homotopy type.

\begin{nlemma}
  Any two spaces $X$ and $Y$ are homotopy equivilent if there exists some other space $Z$ containing both $X$ and $Y$ are deformation retracts.
\end{nlemma}

\begin{ndefi}[Contractible]
  A space that has the homotopy type of a point is contractible.
\end{ndefi}

This amounts to the identity map of this space to be nullhomotopic, that is homotopic to a constant map.

\subsection{Cell Complexes}

To construct a cell complex, let us follow these rules,
\begin{enumerate}
  \item Start with a discrete set $X^0$, whose points are regarded as $0$-cells.
  \item inductively, form the $n$-skeleton $X^n$ from $X^{n-1}$ by attaching $n$-cells $e^n_\a$ via maps $\phi_\a : S^{n-1} \to X^{n-1}$ with a collection of $n$-disks $D^n_\a$ under the identifications $x \sim \phi_\a (x)$ for $x \in \partial D^n_\a$. Thus $X^n = X^{n-1}\coprod_\a e^n_\a$.
  \item We can either stop at a finite amount or continue indefinitely. in the latter, $X$ is given a weak topology, A set $A \subset X$ is open if and only if $A \cap X^n$ is open in $X^n$ for each $n$. This similarly works for closed.
\end{enumerate}

$X$ is a cell complex or a CW complex. If $X = X^n$ then $X$ is said to be finite dimensional and the smallest such $n$ is the dimension of $X$, the maximum dimension of cells of $X$.

\begin{notation}
  $e^n$ denotes an $n$-cell.
\end{notation}

\begin{ndefi}[Characteristic Map]
  Each cell $e^n_\a$ in a cell complex $X$ has a characteristic map $\Phi_\a : D^n_\a \to X$ which extends $\phi_\a$ and is a homeomorphism from the interior of $D^n_\a$ onto $e^n_\a$. Namely, we can take $\phi_\a$ to be the composition, $D^n_\a \emd X^{n-1} \coprod_\a D^n_\a X^n \emd X^n$, where the middle map is the quotient map defining $X^n$.
\end{ndefi}

\begin{ndefi}[Subcomplex]
  A subcomplex of a cell complex $X$ is a closed subspace $A \subset X$ this is a union of cells of $X$.
\end{ndefi}

\begin{ndefi}[CW Pair]
  A complex $X$ with a subcomplex $A$, $(X, A)$, is called a CW pair.
\end{ndefi}
